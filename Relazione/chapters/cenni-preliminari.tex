\chapter{Cenni preliminari}

	\section{Introduzione}
		In questo progetto sono stati implementati cinque storici cifrari a chiave simmetrica. I cifrari implementati sono:
		\begin{itemize}
			\item Shift cipher
			\item Vigenere cipher
			\item Substitution cipher
			\item Affine cipher
			\item OneTimePad cipher
		\end{itemize}
		
	\section{Struttura progetto}
		Le cinque classi principali che implementano i cinque cifrari sono \emph{Shift, Vigenere, Substitution, Affine} e \emph{OneTimePad}.
		
		L'interfaccia \emph{Parser} definisce i metodi che deve implementare la classe che si occuperà di gestire le stringhe rappresentanti i messaggi, le chiavi e quant'altro all'interno del progetto. Viene implementata dalla classe \emph{InputManager}.
		
		L'interfaccia \emph{Mapper} definisce i metodi che deve implementare la classe che si occuperà di gestire le associazioni tra l'alfabeto e la sua cifratura. Viene implementata dalla classe \emph{BiMapper}.
		
		La classe \emph{Constants} contiene delle costanti statiche condivise dalle varie parti del programma e infine la classe \emph{Main} contiene un piccolo main che si limita ad illustrare le varie funzionalità del programma.
		
		Per quanto riguarda i test, sono presenti sette classi di unit-testing (una per ogni cifrario, una per la classe \emph{InputManager} e una per la classe \emph{BiMapper}) e cinque classi di integration-testing per controllare l'effettivo funzionamento dell'interazione tra ognuna delle cinque classi che implementano i cifrari, il parser \emph{InputManager} e il mapper \emph{BiMapper}.