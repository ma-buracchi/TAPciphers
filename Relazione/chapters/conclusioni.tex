\chapter{Conclusioni}
	In questo progetto la cosa che più ha cambiato il mio approccio alla programmazione è stato l'utilizzo di \emph{SonarQube}. Buona parte delle sue segnalazioni erano infatti concetti per me assolutamente nuovi e ho passato molto tempo ad approfondire i vari aspetti legati all'infrazione di tali regole e ai miglioramenti apportati al codice dalla prima stesura.
	
	L'utilizzo della tecnica di \emph{mocking} era per me completamente nuova e non è stato semplice entrare bene nella meccanica. Una volta compresa l'idea fondamentale che c'è sotto (quella di rendere ogni classe indipendente e testabile singolarmente) risulta però abbastanza naturale il suo utilizzo e migliora di tanto la qualità dei testing di una singola classe.
	
	Avevo già implementato una versione simile di questo progetto e, quando ho dovuto utilizzare la libreria \emph{google-collection} che implementa le \emph{HashMap} bidirezionali ho dovuto cercare e scaricare anche altre due dipendenze collegate a questa. Con Maven questo problema viene risolto automaticamente e quindi ho solo dovuto configurare il pom con la dipendenza che effettivamente serviva a me senza preoccuparmi di tutte le dipendenze transitive.
	
	Non utilizzando servizi particolari come DataBase o altro, in un primo momento non avevo previsto l'utilizzo di \emph{Docker}. Questo però si è rivelato fondamentale quando ho iniziato ad utilizzare \emph{SonarQube} che necessita di un proprio server. Grazie a \emph{Docker}, non c'è stato bisogno di scaricare un server di \emph{SonarQube} ma è stato eseguito direttamente all'interno di un container. 