%--------------------------------------------------------------
% - template for the main file of Informatica@Unifi Thesis 
% - based on Classic Thesis Style Copyright (C) 2008 
%   Andr\'e Miede http://www.miede.de   
%--------------------------------------------------------------
% Avoid warning
\RequirePackage{silence}
\WarningFilter{scrbook}{Usage of package `titlesec'}
\WarningFilter{scrbook}{Activating an ugly workaround}
\WarningFilter{scrbook}{\float@addtolist}
\WarningFilter{titlesec}{Non standard sectioning command detected}
%--------------------------------------------------------------
\documentclass[final,oneside,titlepage,fleqn,
	headinclude,12pt,a4paper,BCOR5mm,footinclude]{scrbook}
%--------------------------------------------------------------
\newcommand{\myItalianTitle}{CIPHERS: IMPLEMENTAZIONE DI ALCUNI CIFRARI STORICI\xspace} 
\newcommand{\myDegree}{Corso di Laurea Magistrale in Informatica\xspace}
\newcommand{\myName}{Marco Buracchi\xspace}
\newcommand{\myProf}{Prof. Lorenzo Bettini\xspace}
\newcommand{\myFaculty}{
	Scuola di Scienze Matematiche, Fisiche e Naturali\xspace}
\newcommand{\myUni}{\protect{
	Universit\`A degli Studi di Firenze}\xspace}
\newcommand{\myLocation}{Firenze\xspace}
\newcommand{\myTime}{Anno Accademico 2016-2017\xspace} 
%--------------------------------------------------------------
\usepackage{dia-classicthesis-ldpkg}
%--------------------------------------------------------------
% Options for classicthesis.sty:
% tocaligned eulerchapternumbers drafting linedheaders 
% listsseparated subfig nochapters beramono eulermath parts 
% minionpro pdfspacing
\usepackage[eulerchapternumbers,linedheaders,subfig,beramono,eulermath,parts]{classicthesis}
%--------------------------------------------------------------
\newcommand{\svar}{\textbf{S}}
\newcommand{\ssvar}{\textbf{SS}}
\newcommand{\sssvar}{\textbf{SSS}}
\newcommand{\ssssvar}{\textbf{SSSS}}
\newcommand{\sssssvar}{\textbf{SSSSS}}
\newcommand{\ssssssvar}{\textbf{SSSSSS}}
\newcommand{\sssssssvar}{\textbf{SSSSSSS}}
\newcommand{\kvar}{\textbf{K}}
\newcommand{\ivar}{\textbf{I}}
\newcommand{\approssima}{\approx_{b}}
\renewcommand{\epsilon}{\varepsilon}	%convenzione matematica
\renewcommand{\theta}{\vartheta}		%convenzione matematica
\renewcommand{\rho}{\varrho}			%convenzione matematica
\renewcommand{\phi}{\varphi}			%convenzione matematica
%--------------------------------------------------------------
\usepackage[italian]{babel}	%sillabazione italiana
\usepackage[utf8]{inputenc}	%accenti direttamente da tastiera
\usepackage[T1]{fontenc}	%stampare accenti
\usepackage[fleqn]{amsmath}	%pacchetto matematico
\usepackage{amssymb}		%simboli matematici
\usepackage{ellipsis}		%spazio dopo punteggiatura			
\usepackage{subfig}			%opzioni per immagini
\usepackage{caption}		%opzioni per didascalie immagini
\usepackage{appendix}		%creare appendice
\usepackage{siunitx}		%unit� di misura sistema internazionale
\usepackage{stmaryrd}		%altri simboli matematici
\usepackage{tikz}			%disegnare
\usepackage[object=vectorian]{pgfornament}%ornamenti pagine
\usepackage{listings}		%inserire codice
\usepackage{mathpartir}		%regole di inferenza
%--------------------------------------------------------------
\newlength{\abcd} % for ab..z string length calculation
% how all the floats will be aligned
\newcommand{\myfloatalign}{\centering} 
\setlength{\extrarowheight}{3pt} % increase table row height
\captionsetup{format=hang,font=small}
%--------------------------------------------------------------
% Layout setting
%--------------------------------------------------------------
\usepackage{geometry}
\geometry{
	a4paper,
	ignoremp,
	bindingoffset = 1cm, 
	textwidth     = 13.5cm,
	textheight    = 21.5cm,
	lmargin       = 3.5cm, % left margin
	tmargin       = 4cm    % top margin 
}

\lstset{
  	frame=tb,
	language=Matlab,
  	aboveskip=3mm,
  	belowskip=3mm,
  	showstringspaces=false,
  	columns=flexible,
  	basicstyle={\small\ttfamily},
  	numbers=none,
  	breaklines=true,
  	breakatwhitespace=true,
  	tabsize=3
}

\newcommand{\sectionline}{
	\begin{center}
		\resizebox{0.5\linewidth}{1ex}{
			\begin{tikzpicture}
				\node  (C) at (0,0) {};
				\node (D) at (9,0) {};
				\path (C) to [ornament=83] (D);
			\end{tikzpicture}
		}
	\end{center}
}
%--------------------------------------------------------------
\begin{document}
\frenchspacing
\raggedbottom
%\pagenumbering{roman}
\pagestyle{plain}
%--------------------------------------------------------------
% Frontmatter
%--------------------------------------------------------------
%--------------------------------------------------------------
% titlepage.tex (use thesis.tex as main file)
%--------------------------------------------------------------
\begin{titlepage}
	\begin{center}
   	\large
      \hfill
      \vfill
      \begingroup
         \includegraphics[scale=0.15]{logo/LOGO}\\
			\spacedallcaps{\myUni} \\ 
			\myFaculty \\
			\myDegree \\ 
			\vspace{0.5cm}
         \vspace{0.5cm}    
         Corso di Tecniche Avanzate di Programmazione
      \endgroup 
      \vfill 
      \begingroup
      	\color{Maroon}\spacedallcaps{\myItalianTitle} \\ $\ $\\
%      	\spacedallcaps{\myEnglishTitle}	
	\bigskip
      \endgroup
      \spacedlowsmallcaps{\myName}
      \vfill 
      \vfill
      \emph{\myProf}
      %Relatore: \emph{Andrea Bondavalli}\\
      %Correlatore: \emph{Andrea Ceccarelli}\\
      \vfill
      \vfill
      \myTime
      \vfill                      
	\end{center}        
\end{titlepage}   
%--------------------------------------------------------------
% back titlepage
%--------------------------------------------------------------
%   \newpage
%	\thispagestyle{empty}
%	\hfill
%	\vfill
%	\noindent\myName: 
%	\textit{\myItalianTitle,} 
%	\myDegree, 
%	%\textcopyright\ 
%	\myTime
%--------------------------------------------------------------
% back titlepage end
%--------------------------------------------------------------
\tableofcontents
\chapter{Cenni preliminari}

	\section{Introduzione}
		In questo progetto sono stati implementati cinque storici cifrari a chiave simmetrica. I cifrari implementati sono:
		\begin{itemize}
			\item Shift cipher (altrimenti noto come cifrario "Giulio Cesare")
			\item Vigenere cipher (l'evoluzione del cifrario di Giulio Cesare)
			\item Substitution cipher
			\item Affine cipher
			\item OneTimePad cipher (il cifrario perfetto)
		\end{itemize}
		
	\section{Struttura progetto}
		Il progetto consta di otto classi. Cinque rappresentano i cinque cifrari, una contiene solamente delle costanti utilizzate più o meno da tutte e cinque le classi principali, una è la classe che implementa il parser che si occupa di gestire i messaggi e le varie stringhe utilizzate dal programma e infine un'interfaccia per le possibili implementazioni di parser differenti.
		
		Tutte le classi sono state create con la metodologia test-driven e testate attraverso unit-testing.
\chapter{Testing}
	\section{Unit testing}
		Andiamo a vedere nel dettaglio i test con i quali sono state implementate le varie classi.
		\subsection{Shift cipher}
			Lo shift cipher è il cifrario più semplice. L'unica cosa che fa è spostare la lettere da cifrare di un certo numero di posizioni avanti nell'alfabeto. Tale numero di posizioni è la chiave del cifrario. Per decodificare il messaggio basterà semplicemente tornare indietro del numero di posizioni indicato dalla chiave. Se per esempio il messaggio è "ciao" e la chiave è 3, il messaggio cifrato sarà "fldr".
			
			Considerando di ottenere solamente stringhe di caratteri cifrabili dal parser (nell'ultima sezione di questo capitolo verrà spiegato il suo utilizzo tramite mocking), le uniche due funzionalità richieste a questo cifrario sono quelle di cifratura e decifratura. Tali funzionalità sono state testate nel caso di uno spostamento di zero posizioni (non ottenendo una cifratura), di una posizione, di ventisei posizioni, di ventisette posizioni e di un numero negativo di posizioni(-1). Come caso limite è stata testata anche la cifratura della stringa vuota (il numero di posizioni è ininfluente dato che la cifratura terminerà immediatamente).
			
			Testando anche la decodifica di tutti questi casi sono state coperte le possibili modalità di cifratura/decifratura di una stringa. 
		\subsection{Mocking}
	\section{Mutation testing}
	\section{Integration testing}
	\section{Test sull'interfaccia}
\chapter{Plug-in utilizzati}
	\section{Maven}
		Maven è stato utilizzato per gestire le varie dipendenze del progetto e la build automatica.
		Le dipendenze necessarie sono: 
		\begin{itemize}
			\item \emph{JUnit}: framework per lo unit-testing
			\item \emph{Guava}: libreria di GOOGLE che fornisce un hashmap bidirezionale
			\item \emph{Mockito}: framework utlizzato sempre per lo unit-testing
			\item \emph{Log4J}: framework per il logging 
		\end{itemize}
		
	\section{Travis}
		Travis viene utilizzato per la continuous integration. Il relativo file di configurazione è stato settato per utilizzare la jdk8 necessaria alla compilazione del progetto e per fornire il risultato della code coverage a coveralls.
		
	\section{Coveralls}
		Coveralls riceve i dati della coverage direttamente da Travis.
		
	\section{Docker}
		L'utilizzo principale di Docker è in realtà usare docker compose per gestire il server di SonarQube senza doverlo effettivamente scaricare. 
		
	\section{Sonarqube}
		Sonarqube viene utilizzato tramite docker compose. E' stata scaricata l'ultima versione, attivando tutte le regole disponibili per Java ad esclusione delle seguenti:
		\begin{itemize}
			\item squid
		\end{itemize}
\include{004}
\begin{appendices}
	\chapter{Svolgimento completo esercizio 5.4}
\end{appendices}
\pagestyle{scrheadings}
%--------------------------------------------------------------
% Mainmatter
%--------------------------------------------------------------

		
%--------------------------------------------------------------
\end{document}
%--------------------------------------------------------------
