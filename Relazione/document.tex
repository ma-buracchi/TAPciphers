%--------------------------------------------------------------
% - template for the main file of Informatica@Unifi Thesis 
% - based on Classic Thesis Style Copyright (C) 2008 
%   Andr\'e Miede http://www.miede.de   
%--------------------------------------------------------------
% Avoid warning
\RequirePackage{silence}
\WarningFilter{scrbook}{Usage of package `titlesec'}
\WarningFilter{scrbook}{Activating an ugly workaround}
\WarningFilter{scrbook}{\float@addtolist}
\WarningFilter{titlesec}{Non standard sectioning command detected}
%--------------------------------------------------------------
\documentclass[final,oneside,titlepage,fleqn,
	headinclude,10pt,a4paper,BCOR5mm,footinclude]{scrbook}
%--------------------------------------------------------------
\newcommand{\myItalianTitle}{CIPHERS: IMPLEMENTAZIONE DI ALCUNI CIFRARI STORICI\xspace} 
\newcommand{\myDegree}{Corso di Laurea Magistrale in Informatica\xspace}
\newcommand{\myName}{Marco Buracchi\xspace}
\newcommand{\myProf}{Prof. Lorenzo Bettini\xspace}
\newcommand{\myFaculty}{
	Scuola di Scienze Matematiche, Fisiche e Naturali\xspace}
\newcommand{\myUni}{\protect{
	Universit\`A degli Studi di Firenze}\xspace}
\newcommand{\myLocation}{Firenze\xspace}
\newcommand{\myTime}{Anno Accademico 2016-2017\xspace} 
%--------------------------------------------------------------
\usepackage{dia-classicthesis-ldpkg}
%--------------------------------------------------------------
% Options for classicthesis.sty:
% tocaligned eulerchapternumbers drafting linedheaders 
% listsseparated subfig nochapters beramono eulermath parts 
% minionpro pdfspacing
\usepackage[eulerchapternumbers,linedheaders,subfig,beramono,eulermath,parts]{classicthesis}
%--------------------------------------------------------------
%--------------------------------------------------------------
\usepackage[italian]{babel}	%sillabazione italiana
\usepackage[utf8]{inputenc}	%accenti direttamente da tastiera
\usepackage[T1]{fontenc}	%stampare accenti
\usepackage{ellipsis}		%spazio dopo punteggiatura			
\usepackage{subfig}			%opzioni per immagini
\usepackage{caption}		%opzioni per didascalie immagini
\usepackage{appendix}		%creare appendice
\usepackage{listings}		%inserire codice
\usepackage{color}
\definecolor{gray}{rgb}{0.4,0.4,0.4}
\definecolor{darkblue}{rgb}{0.0,0.0,0.6}
\definecolor{cyan}{rgb}{0.0,0.6,0.6}
%--------------------------------------------------------------
\newlength{\abcd} % for ab..z string length calculation
% how all the floats will be aligned
\newcommand{\myfloatalign}{\centering} 
\setlength{\extrarowheight}{3pt} % increase table row height
\captionsetup{format=hang,font=small}
%--------------------------------------------------------------
% Layout setting
%--------------------------------------------------------------
\usepackage{geometry}
\geometry{
	a4paper,
	ignoremp,
	bindingoffset = 1cm, 
	textwidth     = 13.5cm,
	textheight    = 21.5cm,
	lmargin       = 3.5cm, % left margin
	tmargin       = 4cm    % top margin 
}

\lstset{
	basicstyle=\ttfamily,
	columns=fullflexible,
	showstringspaces=false,
	commentstyle=\color{gray}\upshape
}

\lstdefinelanguage{XML}
{
	morestring=[b]",
	morestring=[s]{>}{<},
	morecomment=[s]{<?}{?>},
	stringstyle=\color{black},
	identifierstyle=\color{darkblue},
	keywordstyle=\color{cyan},
	morekeywords={xmlns,version,type}% list your attributes here
}

\newcommand{\sectionline}{
	\begin{center}
		\resizebox{0.5\linewidth}{1ex}{
			\begin{tikzpicture}
				\node  (C) at (0,0) {};
				\node (D) at (9,0) {};
				\path (C) to [ornament=83] (D);
			\end{tikzpicture}
		}
	\end{center}
}
%--------------------------------------------------------------
\begin{document}
\frenchspacing
\raggedbottom
%\pagenumbering{roman}
\pagestyle{plain}
%--------------------------------------------------------------
% Frontmatter
%--------------------------------------------------------------
%--------------------------------------------------------------
% titlepage.tex (use thesis.tex as main file)
%--------------------------------------------------------------
\begin{titlepage}
	\begin{center}
   	\large
      \hfill
      \vfill
      \begingroup
         \includegraphics[scale=0.15]{logo/LOGO}\\
			\spacedallcaps{\myUni} \\ 
			\myFaculty \\
			\myDegree \\ 
			\vspace{0.5cm}
         \vspace{0.5cm}    
         Corso di Tecniche Avanzate di Programmazione
      \endgroup 
      \vfill 
      \begingroup
      	\color{Maroon}\spacedallcaps{\myItalianTitle} \\ $\ $\\
%      	\spacedallcaps{\myEnglishTitle}	
	\bigskip
      \endgroup
      \spacedlowsmallcaps{\myName}
      \vfill 
      \vfill
      \emph{\myProf}
      %Relatore: \emph{Andrea Bondavalli}\\
      %Correlatore: \emph{Andrea Ceccarelli}\\
      \vfill
      \vfill
      \myTime
      \vfill                      
	\end{center}        
\end{titlepage}   
%--------------------------------------------------------------
% back titlepage
%--------------------------------------------------------------
%   \newpage
%	\thispagestyle{empty}
%	\hfill
%	\vfill
%	\noindent\myName: 
%	\textit{\myItalianTitle,} 
%	\myDegree, 
%	%\textcopyright\ 
%	\myTime
%--------------------------------------------------------------
% back titlepage end
%--------------------------------------------------------------
\tableofcontents
\chapter{Cenni preliminari}

	\section{Introduzione}
		In questo progetto sono stati implementati cinque storici cifrari a chiave simmetrica. I cifrari implementati sono:
		\begin{itemize}
			\item Shift cipher
			\item Vigenere cipher
			\item Substitution cipher
			\item Affine cipher
			\item OneTimePad cipher
		\end{itemize}
		
	\section{Struttura progetto}
		Le cinque classi principali che implementano i cinque cifrari sono \emph{Shift, Vigenere, Substitution, Affine} e \emph{OneTimePad}.
		
		L'interfaccia \emph{Parser} definisce i metodi che deve implementare la classe che si occuperà di gestire le stringhe rappresentanti i messaggi, le chiavi e quant'altro all'interno del progetto.
		
		La classe \emph{InputManager} fornisce un'implementazione di tale interfaccia mentre la classe \emph{Constants} contiene delle variabili condivise.
		
		Infine la classe \emph{Main} contiene un piccolo main utile a spiegare le varie funzionalità del programma.
		
		Per quanto riguarda i test, sono presenti sei classi di unit-testing (una per ogni cifrario e una per l'\emph{InputManager}) e cinque classi di integration-testing per controllare l'effettivo funzionamento dell'interazione tra ognuna delle cinque classi che implementano i cifrari e il parser \emph{InputManager}.
\chapter{Strumenti utilizzati}

	\section{Maven}
		Maven è stato utilizzato per gestire le varie dipendenze del progetto e la build automatica.
		Le dipendenze necessarie al programma e quindi inserite nel fatJar sono: 
		\begin{itemize}
			\item \emph{Guava}: libreria di GOOGLE che fornisce un hashmap bidirezionale
			\item \emph{Log4J}: framework per il logging (non realmente necessario ma utilizzato dal Main per stampare messaggi a schermo)
		\end{itemize}
		
	\section{GitHub}
		Il progetto è raggiungibile a questo link https://github.com/ma-buracchi/TAPciphers. 
		
		Inizialmente, per prendere dimestichezza con il processo di branching e di pull request, le classi Shift, Substitution e InputManager sono state create appunto su tre branch diversi e riportate sul ramo master tramite una merge di tre pull request. Essendo comunque un progetto svolto da una singola persona, le rimanenti classi sono state sviluppate direttamente sul ramo master. Un'eccezione è stata fatta per questa relazione, sviluppata su un altro ramo e successivamente mergiata sul ramo master sempre tramite pull request.	
		
	\section{Travis}
		Travis viene utilizzato per la continuous integration. Il relativo file di configurazione è stato settato per utilizzare la jdk8 necessaria alla compilazione del progetto e per fornire il risultato della code coverage a coveralls.
		
		Una build automatica viene lanciata ad ogni push effettuata sul repository GitHub contenente il progetto.
		
		Il progetto è raggiungibile al seguente link: https://travis-ci.org/ma-buracchi/TAPciphers
		
	\section{Coveralls}
		Coveralls riceve i dati della coverage direttamente da Travis.
		
		\begin{figure}[h]
			\centering
			\includegraphics[scale=0.2]{img/coverall}
			\caption{Risultati di Coveralls}
			\label{fig:coveralls}
		\end{figure}
				
		I risultati della code coverage sono visibili al link https://coveralls.io/github/ma-buracchi/TAPciphers.
		
	\section{Docker}
		L'utilizzo di Docker è quello di usare docker-compose per gestire il server di SonarQube senza doverlo effettivamente scaricare. 
		
	\section{Sonarqube}
		Sonarqube viene utilizzato tramite docker compose. E' stata scaricata l'ultima versione, attivando tutte le 395 regole disponibili per Java ad esclusione di quelle presenti in figura \ref{fig:sonarules}
		
		\begin{figure}[h]
			\centering
			\includegraphics[scale=0.4]{img/sonarqube_rules}
			\caption{Regole disattivate}
			\label{fig:sonarules}
		\end{figure}
		
		Sono state escluse 
		\begin{itemize}
			\item le regole deprecate
			\item le regole riguardanti JavaDoc che non viene utilizzato in quanto questo progetto è pensato per un utilizzo personale e non per essere divulgato
			\item le regole che contestano l'assenza delle informazioni di copyright e di licenza (come sopra)
			\item alcune regole ambigue di formattazione come ad esempio quella che impone di mettere la parentesi graffa aperta di inizio blocco in una nuova linea che contrasta con quella che suggerisce di metterla sulla stessa riga (e con la formattazione di \emph{Eclipse})
		\end{itemize}
		
		Il risultato della scansione utilizzando questi parametri è illustrato in figura \ref{fig:sonaresult}
		
		\begin{figure}[h]
			\centering
			\includegraphics[scale=0.3]{img/sonaresult}
			\caption{Risultato della scansione}
			\label{fig:sonaresult}
		\end{figure}
		
\chapter{Testing}

	\section{Unit testing}
		Per tutti i test viene supposto che le stringhe da cifrare siano state processate dal parser e quindi che contengano solamente caratteri effettivamente cifrabili. Per eliminare la dipendenza da classi concrete, nei vari unit-test, un'istanza del parser e, dove necessario, una del mapper viene simulata tramite \emph{mocking} e \emph{stubbing} dei metodi. Tale metodo verrà approfondito più avanti.
		
		Andiamo a vedere nel dettaglio i test con i quali sono state implementate le varie classi.
		
		\subsection{Shift cipher}
			Lo \emph{shift cipher} è il cifrario più semplice. L'unica cosa che fa è spostare la lettere da cifrare di un certo numero di posizioni avanti (o indietro) nell'alfabeto. Tale numero di posizioni è la chiave del cifrario. Per decifrare il messaggio basterà semplicemente tornare indietro (o avanti) del numero di posizioni indicato dalla chiave. Se per esempio il messaggio è \emph{ciao} e la chiave è 3, il messaggio cifrato sarà \emph{fldr}.
			
			Le uniche due funzionalità richieste a questo cifrario sono quelle di cifratura e decifratura. Tali funzionalità sono state testate nel caso di uno spostamento di zero posizioni (non ottenendo una cifratura), di una posizione, di ventisei posizioni, di ventisette posizioni e di un numero negativo di posizioni(-1). Come caso limite è stata testata anche la cifratura della stringa vuota (il numero di posizioni è ininfluente dato che la cifratura terminerà immediatamente).
			
			Testando anche la decifratura di tutti questi casi sono state coperte le possibili modalità di cifratura/decifratura di una stringa.
			
		\subsection{Affine cipher}
			L'Affine cipher è una generalizzazione dello shift cipher. Questo cifrario ha come chiave due parametri \emph{a} e \emph{b} opportunamente scelti secondo principi di algebra modulare. La i-esima lettera P[i] del messaggio in chiaro verrà cifrata in (P[i] * \emph{a} + \emph{b}) modulo 26. Se viene scelto il parametro \emph{a} = 1 otteniamo in tutto e per tutto uno shift cipher.
			
			I comportamenti da testare sono la cifratura, la decifratura e la giusta scelta del  parametro \emph{a} che deve essere coprimo con 26.
			
			Questo ultimo comportamento viene testato dando in input 2 (non coprimo con 26) come coefficiente \emph{a} e aspettandosi una IllegalArgumentException.
			
			La cifratura e la decifratura vengono testate con stringhe di lunghezza variabile per cercare di coprire il maggior numero di combinazioni possibili.
			
		\subsection{Vigenere cipher}
			Il \emph{cifrario di Vigenère} è un'altra variante dello shift cipher che invece di utilizzare come chiave un numero fisso, utilizza una parola che viene ripetuta tante volte fino al raggiungimento della lunghezza del messaggio che si vuole cifrare. Ogni lettera verrà poi traslata del numero di posizioni corrispondenti alla posizione nell'alfabeto della corrispondente lettera della chiave.
			
			Se per esempio abbiamo un certo messaggio P da cifrare con una chiave K, allora come prima cosa verrà ripetuta K tante volte fino a che la sua lunghezza non raggiunge quella di P e poi ogni i-esima lettera P[i] del messaggio sarà cifrata in P[i]+K[i] modulo 26.
			
			Cifrare il messaggio \emph{testmessage} con la chiave \emph{test} comporterà tre ripetizioni della chiave (che quindi diventerà \emph{testtesttest}). Conseguentemente la prima lettera del messaggio (\emph{t}) verrà spostata di venti posizioni (verrà cifrata con la prima lettera della chiave che è \emph{t} ed è la ventesima lettera dell'alfabeto), la seconda di cinque e così via.
			
			I comportamenti da testare sono la cifratura, la decifratura e il prolungamento della chiave.
			
			Cifratura e decifratura sono testati con stringhe di lunghezze diverse, sempre per coprire il maggior numero di combinazioni possibili.
			
			L'estensione della chiave viene testata fornendo la stringa \emph{test} come chiave, facendo cifrare il messaggio \emph{testmessage} e verificando che sia stata trasformata in \emph{testtesttest}.
			
			Sono presenti anche tre test che controllano l'inserimento di una chiave illegale. Questi test sono stati inseriti per chiarire meglio il comportamento di questa classe ma sarebbero inutili in quanto questo controllo viene in realtà fatto dal parser.
			
		\subsection{OneTimePad cipher}
			Il cifrario OneTimePad è una specializzazione del cifrario di Vigenère che richiede come chiave una stringa binaria della lunghezza almeno pari alla conversione in stringa binaria del messaggio da cifrare. Questo cifrario viene anche chiamato \emph{cifrario perfetto} ed è l'unico sistema crittografico la cui sicurezza sia comprovata da una dimostrazione matematica nel 1949 da \emph{Claude Shannon}.
			
			Questo cifrario è un po' più complesso dei precedenti in quanto prevede una conversione del messaggio da cifrare in una stringa binaria. Una volta convertita il messaggio cifrato si otterrà semplicemente operando uno XOR tra le due stringhe binarie.
			
			I comportamenti da testare sono quindi molteplici.
			
			Il primo comportamento testato è quello di creazione della chiave (che avviene con un generatore casuale). Questo comportamento viene testato confrontando la lunghezza della stringa data in input e la lunghezza della chiave restituita.
			
			Successivamente viene testata la capacità di convertire un messaggio nella relativa stringa binaria. Per questa conversione viene utilizzata la codifica di Huffmann dell'alfabeto inglese.
			
			Viene testata anche l'operazione inversa, cioè quella di riconvertire una stringa binaria ottenuta dalla codifica precedente in una stringa di testo. Questo permetterà di ottenere un messaggio decifrato leggibile.
			
			Come ultima cosa, vengono testate la cifratura e la decifratura.
			
			Tutti questi controlli, come per i casi precedenti, vengono ripetuti su stringhe di lunghezza differente per coprire ogni caso possibile.
			
		\subsection{Substitution cipher}
			Anche il substitution cipher è un cifrario abbastanza semplice che utilizza come chiave di cifratura una permutazione dell'alfabeto. Se per esempio la permutazione utilizzata è \emph{qwertyuiopasdfghjklzxcvbnm} la lettera \emph{a} verrà sostituita con la \emph{q}, la \emph{b} con la \emph{w} e così via fino a scambiare la \emph{z} con la \emph{m}.
			
			I comportamenti importanti da testare per questo cifrario sono ovviamente la cifratura e la decifratura ma anche il corretto settaggio della permutazione dell'alfabeto da utilizzare come chiave.
			
			Cifratura e decifratura vengono testate come sempre con stringhe di lunghezze diverse. La permutazione utilizzata è appunto quella specificata nell'esempio iniziale.
			
			Come prima, i test riguardanti il controllo della permutazione di cifratura in realtà non sarebbero necessari in quanto già presenti nei test del parser (che è quello che si occupa di questo controllo) ma sono stati aggiunti per specificare meglio il comportamento che deve avere la classe in caso di inserimento di una permutazione scorretta (troppo corta, troppo lunga o con caratteri non alfabetici).
			
		\subsection{InputManager}
			Questa è la classe concreta che implementa l'interfaccia \emph{Parser}. Il suo scopo è gestire le varie stringhe passate ai cifrari restituendo solamente caratteri cifrabili.
			
			Il metodo \emph{process} riceve come parametro una stringa, rimuove tutti i caratteri che non sono lettere e converte tutte le lettere maiuscole in minuscole.Questo metodo viene testato passando come argomento una stringa già nella forma voluta, una con un carattere non alfabetico, una con uno spazio, una con delle maiuscole e una vuota.
			
			Il metodo \emph{checkAlphabet} riceve sempre una stringa come parametro e controlla che sia formata da esattamente ventisei lettere. Se sono meno, più o sono presenti caratteri non alfabetici restituisce un'eccezione.Anche questo metodo viene testato cercando di simulare ogni possibile input quindi con una stringa corretta, una più corta, una più lunga, una con un carattere non alfabetico e una con lettere maiuscole (che non solleva eccezione).
			
			Il metodo \emph{checkKey} riceve una stringa e un intero \emph{l} e si occupa di verificare che la stringa ricevuta sia una stringa binaria di lunghezza \emph{l}. Questo metodo viene testato prima passando una stringa binaria e la lunghezza corrispondente e poi aspettandosi un'eccezione quando gli vengono passate prima una stringa contenente un carattere diverso da 0 o 1 e successivamente un intero maggiore della lunghezza della stringa.
			
	\section{Mocking}
		Per eliminare la dipendenza da altre classi concrete che potrebbero disturbare lo unit-testing, nelle cinque classi che implementano i cifrari, viene utilizzato un \emph{mock} dell'interfaccia parser.	Ognuna di queste cinque classi utilizza infatti un oggetto di questo tipo per effettuare vari controlli sulle stringhe fornite in input.
		
		Lo stubbing dei metodi usati da vari test-case viene effettuato direttamente nel metodo setup annotato come \emph{@Before} mentre quelli usati solamente da un singolo test-case vengono effettuati all'interno del relativo test-case.
		
	\section{Mutation testing}
		Per il mutation testing è stato utilizzato PIT. Sono stati attivati gli strong mutator e il risultato del testing è visibile in figura \ref{fig:pit}
		
		\begin{figure}[h]
			\centering
			\includegraphics[scale=0.5]{img/PIT}
			\caption{Risultati del mutation testing}
			\label{fig:pit}
		\end{figure}
		
		Dei 121 mutanti creati solamente tre sopravvivono. Questo però non viene ritenuto un problema perché i mutanti sopravvissuti non compromettono la funzionalità del programma. Il comportamento mutato è infatti il controllo del fatto che la lunghezza della chiave nel cifrario di Vigenère sia almeno quanto la lunghezza del messaggio che si vuole cifrare. Nel codice è stato utilizzato un < ma ovviamente, visto che la lunghezza della chiave non deve necessariamente essere uguale a quella del messaggio ma può essere superiore, cambiare questo controllo con un <= non cambia la funzionalità del metodo e non deve neanche generare un errore.
		
		L'unico effetto riscontrato sarà un prolungamento inutile della chiave che però non porta alcun problema.
		
	\section{Integration testing}
\chapter{Conclusioni}
	In questo progetto la cosa che più ha cambiato il mio approccio alla programmazione è stato l'utilizzo di \emph{SonarQube}. Buona parte delle sue segnalazioni erano infatti concetti per me assolutamente nuovi e ho passato molto tempo ad approfondire i vari aspetti legati all'infrazione di tali regole e ai miglioramenti apportati al codice dalla prima stesura.
	
	L'utilizzo della tecnica di \emph{mocking} era per me completamente nuova e non è stato semplice entrare bene nella meccanica. Una volta compresa l'idea fondamentale che c'è sotto (quella di rendere ogni classe indipendente e testabile singolarmente) risulta però abbastanza naturale il suo utilizzo e migliora di tanto la qualità dei testing di una singola classe.
	
	Avevo già implementato una versione simile di questo progetto e, quando ho dovuto utilizzare la libreria \emph{google-collection} che implementa le \emph{HashMap} bidirezionali ho dovuto cercare e scaricare anche altre due dipendenze collegate a questa. Con Maven questo problema viene risolto automaticamente e quindi ho solo dovuto configurare il pom con la dipendenza che effettivamente serviva a me senza preoccuparmi di tutte le dipendenze transitive.
	
	Non utilizzando servizi particolari come DataBase o altro, in un primo momento non avevo previsto l'utilizzo di \emph{Docker}. Questo però si è rivelato fondamentale quando ho iniziato ad utilizzare \emph{SonarQube} che necessita di un proprio server. Grazie a \emph{Docker}, non c'è stato bisogno di scaricare un server di \emph{SonarQube} ma è stato eseguito direttamente all'interno di un container. 
\pagestyle{scrheadings}
%--------------------------------------------------------------
% Mainmatter
%--------------------------------------------------------------

		
%--------------------------------------------------------------
\end{document}
%--------------------------------------------------------------
