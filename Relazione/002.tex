\chapter{Testing}
	\section{Unit testing}
		Andiamo a vedere nel dettaglio i test con i quali sono state implementate le varie classi.
		\subsection{Shift cipher}
			Lo shift cipher è il cifrario più semplice. L'unica cosa che fa è spostare la lettere da cifrare di un certo numero di posizioni avanti nell'alfabeto. Tale numero di posizioni è la chiave del cifrario. Per decodificare il messaggio basterà semplicemente tornare indietro del numero di posizioni indicato dalla chiave. Se per esempio il messaggio è "ciao" e la chiave è 3, il messaggio cifrato sarà "fldr".
			
			Considerando di ottenere solamente stringhe di caratteri cifrabili dal parser (nell'ultima sezione di questo capitolo verrà spiegato il suo utilizzo tramite mocking), le uniche due funzionalità richieste a questo cifrario sono quelle di cifratura e decifratura. Tali funzionalità sono state testate nel caso di uno spostamento di zero posizioni (non ottenendo una cifratura), di una posizione, di ventisei posizioni, di ventisette posizioni e di un numero negativo di posizioni(-1). Come caso limite è stata testata anche la cifratura della stringa vuota (il numero di posizioni è ininfluente dato che la cifratura terminerà immediatamente).
			
			Testando anche la decodifica di tutti questi casi sono state coperte le possibili modalità di cifratura/decifratura di una stringa. 
		\subsection{Mocking}
	\section{Mutation testing}
	\section{Integration testing}
	\section{Test sull'interfaccia}